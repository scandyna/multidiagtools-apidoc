\hypertarget{guideline_page_getters}{}\section{Propoerty access}\label{guideline_page_getters}
When a member function returns a property that is directly known, the function name should be directly similar to the propoerty name. Example\+: Some\+Object\+::is\+Null() .

If a member function needs some processing before returning the requested value, it should be prefixed get. For example, we suppose that getting available tables in a database needs to query the database, the function should be named like Some\+Object\+::get\+Available\+Tables(\+T db).\hypertarget{guideline_page_return_value}{}\section{Return value}\label{guideline_page_return_value}
When a function fails, it should be possible to get the reason why it failed. This is done by using the \hyperlink{class_mdt_1_1_error}{Mdt\+::\+Error} class.

Functionnal classes should store a \hyperlink{class_mdt_1_1_error}{Mdt\+::\+Error} as member and provide last\+Error() function. Their member functions should follow these rules\+:
\begin{DoxyItemize}
\item If the function can fail, but returns no data, it should return a bool and store the error in last\+Error.
\item If the function can fail, and returns data, it should return a mdt\+Expected. For coherence reason, the error should also be stored in last\+Error.
\item If the function can fail, and it is static, it should return a \hyperlink{class_mdt_1_1_expected}{Mdt\+::\+Expected}.
\end{DoxyItemize}

Value classes should not have functions that can fail, because they do not, for example, read files. If it occurs that function can fail, they should follow these rules\+:
\begin{DoxyItemize}
\item If the function can fail, it should return a \hyperlink{class_mdt_1_1_expected}{Mdt\+::\+Expected}.
\end{DoxyItemize}\hypertarget{guideline_page_error_handlig}{}\section{Error handling}\label{guideline_page_error_handlig}
\hypertarget{guideline_page_error_handlig_dialog}{}\subsection{Dialogs}\label{guideline_page_error_handlig_dialog}
A dialog is a top level widget that is displayed to the user. When a error occurs, the dialog should also display it to the user.

If a public function can fail (for example setting a file), it should display the error to the user and return false, so the caller knows that something failed and he adapt his flow. \begin{DoxyNote}{Note}
If something could fail, the dialog should inform the user, and should not be acceptable.
\end{DoxyNote}
\hypertarget{guideline_page_ui_files}{}\section{Ui files when using Qt\+Designer}\label{guideline_page_ui_files}
One of the possible way to use headers generated by uic, is to inherit from the base class and from the class generated by uic. This option requires that ui\+\_\+$\ast$.h is included in the header file of the class. Because this header file is generated during compilation, it will be somwhere in the build tree. For this library, this issue has to be solved, and is by setting correct C\+Make options. But, the problem will then happen again to the user, which has no access to this ui\+\_\+$\ast$h file.

To solve this issue, the Ui class should be forward declared in the header, used as member of the class, using a unique\+\_\+ptr. The ui\+\_\+$\ast$h file can also be included in the $\ast$.cpp file. See Mdt\+::\+Item\+Editor\+::\+Standard\+Window to see a example of this usage.\hypertarget{guideline_page_translations}{}\section{Translations}\label{guideline_page_translations}
\hypertarget{guideline_page_translations_tr}{}\subsection{About using tr()}\label{guideline_page_translations_tr}
When a class is a subclass of Q\+Object, simply using tr() works fine.

For a class that is not a subclass of Q\+Object, Qt recommends using Q\+Core\+Application\+::translate() and provide the class name as context, so that Qt Linguist can display it the proper way to the translator. Previously, many classes in \hyperlink{namespace_mdt}{Mdt} libraries have added a tr() static member function, that simply called Q\+Object\+::tr(). This is bad, because Qt Linguist also displayed all strings as Q\+Object context. The recommanded solution is to use Q\+\_\+\+D\+E\+C\+L\+A\+R\+E\+\_\+\+T\+R\+\_\+\+F\+U\+N\+C\+T\+I\+O\+N\+S() provided in Q\+Core\+Application For example, in My\+Class.\+h\+: 
\begin{DoxyCode}
\textcolor{preprocessor}{#include <QCoreApplication>}

\textcolor{keyword}{class }MyClass
\{
  Q\_DECLARE\_TR\_FUNCTIONS(MyClass)

 public:

  MyClass();
\};
\end{DoxyCode}


Note that Qt Linguist trows error messages when a class uses tr() and it does not provide the Q\+\_\+\+O\+B\+J\+E\+CT macro, despite the fact that translation file are generated and works. 